\documentclass[pdftex, 12pt, oneside]{article}

\usepackage[paperwidth=8.27in, paperheight=11.69in]{geometry}
\usepackage{graphicx}
\usepackage[english]{babel}
\usepackage{enumerate}
\usepackage{float}
\usepackage{gensymb}
\usepackage{listings}

\newcommand{\HRule}{\rule{\linewidth}{0.5mm}}

\begin{document}

\textbf{\large TUTORIAL ZKOSS - HI }
\\[1cm]
\textbf{Priyanto Tamami}\\
tamami.oka@gmail.com\\
github.com/tamami

\HRule

\section{PENDAHULUAN}

Dengan berkembangnya dunia internet sekarang ini, kebutuhan akan sebuah sistem informasi berbasis \textit{web} semakin banyak diperlukan. Dari kebutuhan hanya sekedar menampilkan informasi perusahaan, sampai kepada pengolahan data entry sehingga menghasilkan laporan-laporan yang dapat diakses melalui sebuah \textit{website}.

Perkembangan jumlah pengguna internet yang semakin meningkat, pun demikian dari sisi penyedianya, entah yang menawarkan jasa \textit{hosting}, jasa \textit{cloud}, sampai dengan jasa untuk membangun sebuah aplikasi web.

Perkembangan teknologi untuk membangun aplikasi web pun tidak kalah berkembang, banyak \textit{framework} bermunculan dalam beberapa bahasa pemrograman, yang salah satu tujuannya yaitu mempercepat waktu dalam membangun sebuah aplikasi berbasis web. ZKOSS adalah salah satunya.

\section{ISI}

\textit{Framework} ZKOSS ini memang bukan sepenuhnya gratis, ada lisensi berbayarnya juga untuk beberapa \textit{tools}, jadi pembahasan kali ini kita akan coba yang versi gratisnya saja, yaitu versi komunitas.

\textit{Framework} ini mengusung konsep pembangunan aplikasi dengan model MVVM (Model-View-ViewModel), walau secara kasar tidak akan terlihat bedanya dengan konsep MVC (Model-View-Controller), dan ZKOSS pun sebetulnya mampu mengimplementasikan konsep MVC yang sudah terkenal.

Mari kita buat aplikasi sederhana untuk mengatakan hai pada dunia.

\section{PENUTUP}

\end{document}